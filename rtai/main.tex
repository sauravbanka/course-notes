\documentclass[11pt]{article}
\usepackage[margin,header,sans,titlepage,bib]{../lecture_notes}


\title{Reliable and Trustworthy AI}
\author{Saurav Banka}
\semester{HS 2025}
\lecturer{Dr. Martin Vechev}
\date{\today}

\begin{document}
\maketitle

\section{Lecture 1: Introduction (17.09.25)}
\subsection{Motivation} % (fold)
\label{sub:motivation}
Traditional ML progress focused on \textbf{standard accuracy} (e.g., ImageNet).  
Deployments in real-world settings often reveal failures:
\begin{itemize}
    \item Distribution shifts $\rightarrow$ performance drop.
    \item Safety-critical failures (autonomous driving, incorrect medical diagnoses).
\end{itemize} 

When deploying models into the real world, we need to know that models are safe, secure, robust, transparent and reliable. The goal of this course is to get a fundamental understanding of privacy and robustness techniques, and a glimpse into the latest research into reliable and trustworthy AI.
% subsection motivation (end)

\subsection{Vertical I: Robustness}
\label{sub:robustness_intro}
Robustness research broadly tackles questions:
\begin{itemize}
	\item What attacks (e.g.\ gradient-based, branch-and-bound) and defenses exist?
	\item Is it possible to certify / prove robustness and performance under perturbations or adversaries?
	\item How to train models that are provably robust?
\end{itemize}

\subsubsection{Why is certifying robustness difficult?}
Consider the use-case of image classification (e.g.\ MRI images) under input perturbations (noise, blur, rotation) in a range that does not affect human ground-truth labels.  
The decision boundary of a neural network is a $(d-1)$-dimensional hypersurface in the input space $\RR^d$.  
The goal is that the model prediction should remain invariant to such perturbations.

Certifying robustness involves:
\begin{enumerate}
	\item \textbf{Precondition $\varphi(x)$:} convex polytope of all possible perturbations of $x$.  
	      \emph{Problem:} prohibitively large to enumerate, especially in higher dimensions or under multiple transformations.
	\item \textbf{Propagation:} push this region through the network layers.  
	      \emph{Problem:} results in many small non-convex shapes. Exact methods (MILP, SMT) are NP-complete and do not scale.
	\item \textbf{Abstraction:} use convex relaxations to approximate the shapes with an enclosing convex polytope.  
	      If the final convex region (post-condition) lies entirely in the correct class, the model is robust.  
	      \emph{Problem:} loose relaxations introduce false positives and reduce provability.
\end{enumerate}

\textbf{Summary:} Tight approximations are precise but expensive; loose approximations are efficient but may miss guarantees.

\subsubsection{Training Certified Models}
Even if robustness could be proven efficiently, unless a network is trained to be provable, it is unlikely to satisfy such specifications.

\textbf{Key idea:} Instead of propagating individual datapoints, propagate convex regions (symbolic inputs).  
Backpropagation then uses symbolic information to reduce the size of post-conditions, encouraging perturbations to cluster closely in representation space.

\paragraph{Objective.}  
Standard training minimizes
\[
    \min_\theta \; \EE \big[ \loss(\theta, x, y) \big].
\]
With a robustness specification, this becomes a min–max optimization:
\[
    \min_\theta \; \EE \Big[ \max_{x' \in \varphi(x)} \loss_{(\varphi, \psi)}(\theta, x', y) \Big].
\]
Interpretation: minimize the worst-case loss over perturbations.  
\emph{Challenge:} This problem is much harder to optimize, may fail to enforce specifications, and often degrades standard accuracy.

\subsubsection{Individual Fairness and Randomized Smoothing}
\textbf{Individual fairness:} If two datapoints are similar in relevant aspects for a task, they should receive similar predictions.  
This is closely related to robustness: perturbations in sensitive attributes should not flip outputs.

\textbf{Randomized smoothing:} An inference-time defense that replaces the classifier with a smoothed version.  
Given an input $x$, add Gaussian noise and average predictions.  
Guarantee: nearby inputs map to the same label with high probability.  
Compared to certified training, randomized smoothing scales better but only provides guarantees for certain robustness properties.


\subsection{Vertical II: Privacy}
\label{sub:privacy_intro}
Privacy in this course is primarily focused on \textbf{data privacy}: preventing leakage of sensitive information from models. Key research questions:

\begin{itemize}
	\item \textbf{Membership inference:} Given a datapoint $x$, can an attacker determine if $x$ was in the training set?
	\item \textbf{Model inversion:} Can an attacker reconstruct a representative example from a given class?
	\item \textbf{Training data extraction:} Can raw training samples be recovered from a model (e.g.\ LLMs regurgitating text)?
	\item \textbf{Private attribute inference:} Can sensitive attributes (age, gender, location) be inferred from user data?  
	\item \textbf{Model stealing:} Recovering model weights or functionality from black-box access.
\end{itemize}

Defenses:
\begin{itemize}
	\item \textbf{Differential Privacy (DP):} Add noise during training (e.g.\ DP-SGD).  
	      Guarantees: presence/absence of one sample does not significantly affect the output distribution.  
	      Trade-off: higher privacy $\leftrightarrow$ lower accuracy. Recent work shows DP-LLMs approaching performance of earlier non-private models.
	\item \textbf{Federated Learning:} Train across distributed devices (e.g.\ smartphones) without centralizing data.  
	      Related to fine-tuning multiple LLMs and merging them securely.
\end{itemize}

Beyond standalone models, in compound systems such as \textbf{agentic AI}, LLMs are integrated with external tools. Safety risks emerge from multi-step workflows (e.g.\ insecure protocols, toxic flows). Ensuring provable safety here remains an open challenge.

% subsection privacy (end)

\subsection{Vertical III: Provenance and Evaluation}
\label{sub:provenance_intro}
Two central themes in this vertical are \textbf{data attribution} (who generated what) and \textbf{evaluation} (how to measure performance fairly).

\subsubsection{Watermarking and Data Attribution}
\begin{itemize}
	\item \textbf{Idea:} Randomly assign tokens a color (expect evenly distributed) and transform the output distribution to skew to a color.
	\item \textbf{Threats:}
		\begin{itemize}
			\item \emph{Stealing:} Approximate the watermark distribution with repeated queries.
			\item \emph{Spoofing:} Generate text that falsely appears to come from a target model.
			\item \emph{Scrubbing:} Remove watermark to conceal that text was model-generated.
		\end{itemize}
\end{itemize}

\subsubsection{Transformations and Safety}
Common LLM transformations (quantization, pruning, distillation, fine-tuning) may impact safety and privacy guarantees.  

\emph{Example:} A model appears normal pre-quantization in benchmarks, etc but after quantization begins inserting hidden adversarial content (e.g.\ advertisements).

\subsubsection{Evaluation and Benchmarks}
\begin{itemize}
	\item \textbf{Benchmarks:}
	      \begin{itemize}
		      \item Closed-form (e.g.\ math problems with unique solutions, math arena).
		      \item Preference-based (e.g.\ LLM Arena where users vote).
	      \end{itemize}
	\item \textbf{Contamination:} Training/test set overlap or task leakage.  
	      Leads to inflated benchmark results.
	\item \textbf{Real-world failures:} Some model releases (e.g.\ K2) were later shown to have flawed evaluations due to contamination or misleading comparisons.
\end{itemize}
\newpage
% subsection provenance (end)
%%%%%%%%%%%%%%%%% END OF LECTURE 1. %%%%%%%%%%%%%%%%% 

\end{document}