\documentclass[11pt]{article}
\usepackage[margin,header,sans,titlepage,bib]{../lecture_notes}

\title{Testing LaTeX sty file}
\author{Saurav Banka}
\semester{HS 2025}
\lecturer{ChatGPT}
\date{\today}
\addbibresource{example.bib}

\begin{document}
\maketitle
\section{Testing Colored Boxes}
\subsection{What are those}

\begin{theorem}\label{thm:oddsquares}
Every odd number is the difference of two squares.
\end{theorem}

\begin{lemma}\label{lem:div}
If $p$ divides $ab$, then $p$ divides $a$ or $b$.
\end{lemma}

\begin{definition}\label{def:prime}
A prime is a number with no nontrivial divisors.
\end{definition}

\begin{example}\label{ex:prime7}
The number $7$ is prime.
\end{example}

\begin{exercise}\label{ex:infprimes}
Prove that there are infinitely many primes.
\end{exercise}

\subsection{What are these}
\begin{note}\label{note:equiv}
This is equivalent to showing that primes cannot be bounded above.
\end{note}


\begin{theorem}\label{thm:murphys}
If something is suspicious...
\begin{proof}
Let $n$ be odd. Then $n = 2k+1$.  
Observe that
\[
( k+1 )^2 - ( k )^2 = (k^2 + 2k + 1) - k^2 = 2k+1 = n.
\]
Thus every odd number is a difference of two squares.
\end{proof}
\end{theorem}


\begin{proposition}[Bayes Rule]
$\Pr(A \mid B) = \dfrac{\Pr(B \mid A) \Pr(A)}{\Pr(B)}$.
\end{proposition}

\begin{corollary}
Every prime $p$ is either $2$ or odd.....
\end{corollary}

\section{Math Shorthands}
Some math macros and operators:

\subsection{Probability}
\[
X \sim \Bin(n,p), \quad Y \sim \Ber(\theta), \quad Z \sim \Nor(0,1).
\]
\[
\theta \sim \BetaDist(\alpha,\beta), \quad 
\lambda \sim \GammaDist(k,\theta), \quad 
N \sim \DistDelta(\mu).
\]
\[
\EE[X], \quad \Var(X), \quad \Cov(X,Y), \quad \Pr(A \cap B).
\]
Conditional expectation: $\Econd{X}{Y}$ \\
Conditional probability: $\Pcond{A}{B}$ \\
Density notation: $\p{x}, \q{z}$

\subsubsection{Gaussian Processes}
Define prior: $f \sim \GP{0}{k(x,x')}$ \\
Posterior mean: $\mu'(x^*) = \kxx{x^*}{X} \big(\Kmat + \sigma^2 I\big)\inv y$


\subsection{Optimization}
\[
\argmin_{x \in \RR^n} f(x), 
\qquad 
\argmax_{\theta \in \Theta} \Pr(D \mid \theta).
\]
\[
\sup_{x \in \RR} g(x), 
\qquad 
\inf_{n \geq 1} a_n.
\]

\subsection{Linear Algebra}
\[
A \vv = \lambda \vv, 
\quad \rank(A), 
\quad \tr(A), 
\quad \Null(A).
\]
Transpose: $A\T$ \\
Inverse: $A\inv$ \\
Half fraction: $\half x^2$ \\
Diagonal matrix: $\diag(\vx)$


\section{Warnings and Drafts}
\begin{danger}
This section might contain misleading or incomplete arguments.
\end{danger}

\begin{ddanger}
Do not attempt this at home: Probability measure $\PP$ over an uncountable set
can behave counterintuitively.
\end{ddanger}

\DRAFT % shows draft notice in footer + date

\section{Algorithms}
\begin{algorithm}
\caption{Sample Pseudocode}
\begin{algorithmic}[1]
\State Initialize $x \gets 0$
\While{$x < 10$}
  \State $x \gets x+1$
\EndWhile
\State \Return $x$
\end{algorithmic}
\end{algorithm}

\section{Cross References}
We can reference earlier results:
\begin{itemize}[itemsep=2pt, topsep=2pt]
  \item See \autoref{thm:oddsquares}.
  \item See \autoref{lem:div}.
  \item See \autoref{def:prime}.
  \item See \autoref{ex:prime7}.
  \item See \autoref{ex:infprimes}.
\end{itemize}


\section{Draft notes and side notes}
This is a sentence. \draftnote{Check this step later.} \\
This needs a picture \sidenote{Draw diagram here}.

\newpage
\appendix
\section{Background on Measure Theory}
Appendix material here.

\section{Background on Information Theory}
No information....

\section{Bibliography Example}
We can cite classic references: see \cite{rudin1987real,cover2006elements}.

\newpage
\printbibliography

\end{document}